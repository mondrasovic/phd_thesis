\section{Multiple Object Tracking}
\label{sec:MultipleObjectTracking}

This section continuously expands the previously started discussion on single object trackers. Our research originally targeted \gls{sot}, especially Siamese single object trackers. The plan to incorporate multiple objects remained only as a hypothesis to explore later since it brings a whole new set of challenges to overcome. However, thanks to our comprehensive survey on Siamese tracking~\cite{ondrasovic2021siamese}, we gained enough background knowledge to quickly absorb the newly emerging body of literature on a specific branch of multiple-object trackers that exploit Siamese architectures. We have to acknowledge that we did not compile a thorough \gls{sota} overview of \gls{mot} for the following reason. Our research is focused on Siamese neural networks, whereas the \gls{mot} is dominated by approaches that utilize detections + linking based on solutions exploiting a wide range of methods, from simple Munkre's algorithm~\cite{munkres1957assignment} through complicated graph formulations~\cite{chen2001motdynamicgraph} to even graph-based convolutional neural networks~\cite{papakis2021gcnnmatch}. Even though there are works that claim the use of Siamese neural networks in \gls{mot}, \egtext{}~\cite{cuan2018deepsiammot}, their utilization serves for the \gls{reid} within the tracking-by-detection philosophy, for which Siamese networks are widely adopted. However, by Siamese tracking, we explicitly mean the type of trackers described in \sectiontext{}~\ref{sec:SingleObjectTracking}. Nevertheless, we did not necessarily need as much background knowledge in \gls{mot} to identify that Siamese-based \gls{mot} is a freshly rising subfield of trackers we should contemplate exploiting due to its direct applicability to traffic analysis. As we will demonstrate, the majority of the best practices from Siamese \gls{sot} have found their use in \gls{mot}, too.

% ##############################################################################
\subsection{Siamese-based Multiple Object Tracking}
\label{ssec:SiameseBasedMultipleObjectTracking}

Shuai~\etal{}~\cite{shuai2020multisiamrcnn} proposed a Siamese-based framework that can simultaneously handle object tracking, detection, and \gls{reid} (\figtext{}~\ref{fig:MultiSiamRCNN}). The unification of all these aspects into a single pipeline is a significant advantage. In addition, the formulation allows the use of any Siamese tracker, which is a great contribution. Although this tracking system follows an inference pipeline similar to other tracking-by-detection systems, the distinction is that it does so based on cues generated by a single network. One important remark made by the authors is that tracking by use of \gls{reid} alone is not robust enough to short-term changes, especially in the presence of partial occlusions. Our experiments will provide corroborating evidence, too.

% ------------------------------------------------------------------------------
\begin{figure}[!t]
    \centerline{\includegraphics[width=\linewidth]{figures/theoretical_foundations/motsiam_trackrcnn_architecture.pdf}}
    \caption[Siamese \gls{mot} with track R-CNN]{Demonstration of how unification of the detection, tracking and \gls{reid} within a single architecture can be achieved. One important aspect that contributes to cost-effectiveness in terms of inference is that features required for all the mentioned tasks share the same backbone, which results in low
        computation and efficient runtime. \externalsrc{\cite{shuai2020multisiamrcnn}}}
    \label{fig:MultiSiamRCNN}
\end{figure}
% ------------------------------------------------------------------------------

From our point of view one trivial, but at the same time, very effective extension of the often-mentioned \gls{siamfc} tracker utilized $n$ exemplars to produce $n$ response maps and, therefore, to perform tracking of $n$ objects simultaneously~\cite{vaquero2021siammt}, aptly dubbed as \gls{siammt} (\figtext{}~\ref{fig:SiamMTArchitecture}). The authors mentioned the incentive to develop their tracker to address the problem with running a costly detector for every frame to produce detections upon which another performance-demanding linking stage is usually executed. This framework was the first to demonstrate the qualities of a purely deep learning-based, end-to-end tracking pipeline capable of tracking multiple arbitrary objects at once. We believe this paradigm of tracking is yet to uncover its full potential.

% ------------------------------------------------------------------------------
\begin{figure}[!t]
    \centering
    \begin{subfigure}[b]{\textwidth}
        \centering
        \includegraphics[width=\textwidth]{figures/theoretical_foundations/siammt_orig.png}
        \caption[]{}
    \end{subfigure}
    \begin{subfigure}[b]{\textwidth}
        \centering
        \includegraphics[width=\textwidth]{figures/theoretical_foundations/siammt_new.png}
        \caption[]{}
    \end{subfigure}
    \caption[\Gls{siammt} architecture]{The inference phase of the \imgpartdesc{a} \gls{siamfc} architecture to see the difference between the \imgpartdesc{b} \gls{siammt} successor. The \gls{siammt} framework first extracts features of the entire frame via the backbone $\varphi$. Subsequently, the obtained features belonging to distinct regions are cropped and resized using the $\tilde{K}$ operator, utilizing \gls{roi}-align operations. Finally, all these features are combined in the traditional cross-correlation way (although slightly adjusted to handle more objects) to produce a multiple-object response map indicating their predicted positions. \externalsrc{\cite{vaquero2021siammt}}}
    \label{fig:SiamMTArchitecture}
\end{figure}
% ------------------------------------------------------------------------------

The endeavor to exploit Siamese neural networks to assess the degree of similarity between two objects has spurred a plethora of proposals combining various mechanisms. Lee~\etal{}~\cite{lee2019motfpsn} combined Siamese similarity learning with \Glspl{fpn} (discussed in \sectiontext{}~\ref{ssec:FeaturePyramidNetworks}). This tracker still follows the path of the tracking-by-detection paradigm, in which the similarity metric between the current detections and existing tracks plays an essential role. In this work, criticism was raised concerning the plain Siamese architectures for not being sufficient for tracking owing to their structural simplicity and lack of motion information. To address the structural simplicity, a \gls{fpsn} was proposed. Then, to overcome the lack of motion information, additional spatiotemporal motion features were added to the \gls{fpsn} module.

% ------------------------------------------------------------------------------
\begin{figure}[!t]
    \centerline{\includegraphics[width=0.6\linewidth]{figures/theoretical_foundations/feature_pyramid_siamese_network.png}}
    \caption[\Gls{fpsn} architecture]{A \gls{cnn} serving as a backbone that adopts Similarity learning as an enhancement to feature aggregation using \gls{fpn}. \externalsrc{\cite{lee2019motfpsn}}}
    \label{fig:FeaturePyramidSiameseNetwork}
\end{figure}
% ------------------------------------------------------------------------------

As a matter of fact, our research ended up working with \gls{siammot}~\cite{shuai2021siammot} (Section~\ref{sec:SiamMOT}) architecture, which we will introduce in great detail later on. It is a multiple-object tracker that practically encompasses some of the best approaches we have discussed so far into an end-to-end framework, such as Siamese tracker (multi-channel cross-correlation), \gls{rpn} head, centerness, feature fusion, and much more.
