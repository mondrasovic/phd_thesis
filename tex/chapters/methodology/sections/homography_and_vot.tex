\section{Homography and Visual Object Tracking}
\label{sec:HomographyAndVisualObjectTracking}

This section is dedicated to one of our experiments that were not completely related to the \gls{vot} itself, yet we achieved an original scientific contribution in this area when exploring certain solutions that could be applicable to object tracking, especially traffic analysis. Even though we did not set out for homography-based object tracking (explained later) due to limitations of available datasets, still we would like to elaborate on our developed approach. The proposed method was fully described as well as scrupulously tested under difficult conditions. We wrote up the whole research process in a paper called \textbf{Homography Ranking Based on Multiple Groups of Point Correspondences}~\cite{ondrasovic2021homography}, published in journal \textbf{Sensors} (web:~\cite{sensors}), under the category of \emph{Physical Sensors}. In what follows, we provide a concise report of our research. For more information, we suggest the reader use our aforementioned article.

\subsection{General Introduction}

One of the fundamental tasks of computer vision is to deal with various image transformations that may improve the outcome of subsequent post-processing phase. One transformation that was of particular interest to our goal of traffic analyis is the perspective transformation. More concretely, a removal of perspective distortion. To achieve this, the so-called homography mapping is often exploited.

Homography is a perspective projection of a plane from one camera view into a different camera view. The perspective projection maps points from a $3$D world onto a $2$D image plane along lines that emanate from a single point~\cite{geetha2013automatic, bousaid2020perspective}. This projection is performed by a $3 \times 3$ invertible transformation matrix called the homography matrix (or just homography) with $8$ \gls{dof}. A general homography matrix may be defined as
\begin{equation}
    \label{eq:HomographyMatrix}
    \H =
    \begin{bmatrix}
        h_{11} & h_{12} & h_{13}\\
        h_{21} & h_{22} & h_{23}\\
        h_{31} & h_{32} & h_{33}
    \end{bmatrix}
\end{equation}
This transformation is used to achieve the mapping between two views of the same plane, since in the pinhole camera model, any two images of the same planar surface are related to each other by the homography~\cite{hartley2003multiple, hartley1997defense}. More specifically, a single vector $\vectt{u}{u_x, u_y, 1}$, representing a warped keypoint in homogeneous coordinates, is mapped onto the rectified keypoint  $\vectt{\tilde{u}}{\tilde{u}_x, \tilde{u}_y, 1}$ by the homography $\H$ using the transformation $s \vect{\tilde{u}} \approx \H \vect{u}$, with $s$ being the scale factor. In its most general form, homography may achieve mapping between various perspectives. However, for our purposes we focused only on producing a view where perspective distortion is absent, i.e., to rectify the image so that it looks as if the camera was in an orthogonal position with respect to the desired plane in the world when taking the picture.

Homography is commonly used for rectification of text document images by generating a fronto-parallel view~\cite{lu2005perspective, miao2006perspective}, image stitching~\cite{adel2014image, gao2011constructing}, video stabilization~\cite{liu2015smooth}, extracting metric information from $2$D images~\cite{zhang2000flexible}, pose estimation~\cite{circularmarkerposeestim}, and for various traffic-related applications, e.g., ground-plane detection~\cite{arrospide2010homography}, and bird's-eye view projection~\cite{luo2010low}.

The primary motivation to explore the possibility of employing homography for visual object tracking was the fact that as long as a static camera is used and few assumptions that we will discuss later hold, the scene may be easily stripped off the effect of the perspective distoriton. Considering this, the incentive to track vehicles visually using a static camera while exploiting a fronto-parallel view over the road seemed like a plausible extension with possible advantages for traffic analysis. Bose et al.~\cite{Bose04groundplane} presented a fully automated technique for both affine and metric rectification of a given ground plane (up to a scale factor) by simply tracking moving objects. The derivation of the necessary constraints for projective transformation between the image and the ground plane was obtained by observing objects that moved at constant velocity in the world for some part of their trajectory. We conjectured that the extra information about the 

A common approach to estimate the homography is to use a set of at least four $2$D point correspondences~\cite{hartley1997defense}. We refer to the points used for establishing the $2$D point correspondences as keypoints. These keypoints may belong to a marker which is an object with a known shape that is either naturally occurring or artificially positioned in the scene. A regular pattern like a chessboard is usually utilized~\cite{zhang2016flexible}. A single marker is identified in the image by multiple independent keypoints that have a direct correspondence to its real shape, thus making a group of point correspondences. However, these correspondences are often noisy and they can introduce errors in the homography estimation. Although $4$ keypoints are satisfactory, often a greater number of keypoints is used, allowing to use optimization to minimize a suitable cost function~\cite{osuna2016multiobjective, mou2013robust}. Then, outlier removal becomes an important step, and algorithms such as RANSAC~\cite{fischler1981random} are usually employed~\cite{osuna2016multiobjective}.

\subsection{Developed Method}

\subsection{Achieved Contribution and Discussion}
