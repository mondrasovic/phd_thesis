\section{Siamese Multi-Object Tracking and Attention}
\label{sec:SiamMOTandAttention}

% ##############################################################################
\subsection{Motivation}

% ##############################################################################
\subsection{Deformable Convolutional Neural Networks}
\label{ssec:DeformableCNNS}

\Glspl{dcnn}~\cite{dai2017dcnn} are gaining popularity and are being applied to numerous sophisticated computer vision tasks, \egtext{}, object segmentation (dense predictions) and object detection (semi-dense predictions). Since object tracking revolves around the same requirements in terms of pixel-wise precision, we contemplated using this advancement as well.

Although \glspl{cnn} (\sectiontext{}~\ref{ssec:ConvolutionalNeuralNetworks} on page~\pageref{ssec:ConvolutionalNeuralNetworks}) are an excellent tool for a plethora of deep learning tasks involving image processing, they are still limited in their capabilities to model a broad range geometric transformation. To address this, practitioners apply a broad range of data augmentation techniques (\egtext{}, rotation, translation, scaling, and shearing) to provide the necessary samples of some particular transformation during the training. However, such an approach is limited to tailor-made transformations that may not cover the entire set of possibilities the model may face in practice.

The first work to learn spatial transformation from the training data in a deep learning fashion is known under the name \glspl{stn}~\cite{jaderberg2016stn}. It warps the feature map via a global parametric transformation such as affine transformation. In the real of convolutional operations, there is the atrous convolution operation~\cite{holschneider1990atrousconv} that enhances the standard convolution by expanding the receptive field while maintaining the same number of parameters by use of greater offsets. However, these offsets are fixed. An obvious successor of this approach is the active convolution~\cite{jeon2017activeconv} that treats convolution offsets as learnable parameters instead of constants. But, in this setting, the learned offsets are shared across different spatial locations. Thus, the most general approach is to determine the offsets at each location independently and then proceed as usual. This is where deformable convolution (see \figtext{}~\ref{fig:StandardVsDeformableCNN}) comes into place, discussed next.

In concrete terms, a $2$D convolution consists of sampling using a regular offset grid $\mset{R}$ defining the receptive field as well as dilation over the input features $\vect{x}$ followed by the summation of the samples values weighted by $\vect{w}$. For example, a standard $3 \times 3$ convolution with dilation $1$ would employ offsets given by
\begin{equation}
    \label{eq:StandardConvolutionOffsetGrid}
    \mset{R} = \cbrackets{
        \rbrackets{-1, -1}, \rbrackets{-1, 0}, \dots, \rbrackets{0, 1}, \rbrackets{1, 1}
    }.
\end{equation}
Then, for each location $\vect{p}_0$ within the output feature map $\vect{y}$ is calculated as
\begin{equation}
    \label{eq:StandardConvolutionOutputCalc}
    \func{\vect{y}}{\vect{p}_0} =
    \sum_{\forall \vect{p}_n \in \mset{R}}
    \func{\vect{w}}{\vect{p}_n} \cdot \func{\vect{x}}{\vect{p}_0 + \vect{p}_n},
\end{equation}
where the locations in $\mset{R}$ are iterated over by $\vect{p}_n$.

Conversely, the deformable convolution extends the standard one by augmenting the original sampling grid $\mset{R}$ with additional offsets $\cbrackets{\Delta \vect{p}_n \ |\ n = 1, \dots, \msetsize{R}}$ (see \figtext{}~\ref{fig:fig:DeformableCNN}). Thus, \eqtext{}~\ref{eq:StandardConvolutionOutputCalc} is reformulated as
\begin{equation}
    \label{eq:DeformableConvolutionOutputCalc}
    \func{\vect{y}}{\vect{p}_0} =
    \sum_{\forall \vect{p}_n \in \mset{R}}
    \func{\vect{w}}{\vect{p}_n} \cdot \func{\vect{x}}{\vect{p}_0 + \vect{p}_n + \Delta \vect{p}_n}.
\end{equation}
However, one needs to keep in mind that the sampling offsets now become fractions and thus have to be handled accordingly. One approach is to employ bilinear interpolation, where the position in the input feature map $\vect{x}$ is determined by
\begin{equation}
    \label{eq:DeformableConvolutionBilinear}
    \func{\vect{x}}{\vect{p}} =
    \sum_{\forall \vect{q}} \func{G}{\vect{q}, \vect{p}} \cdot \func{\vect{x}}{\vect{p}},
\end{equation}
in which $\vect{q}$ enumerates all integral locations and $\func{G}{\cdot}$ represents the  interpolation kernel. The interpolation processing can be efficiently implemented owing to the sparsity. The performance overhead is negligible compared to the reaped benefits of adaptive sampling locations capable of covering very complicated transformations (see \figtext{}~\ref{fig:SamplingLocationsDeformableCNN}).

% ------------------------------------------------------------------------------
\begin{figure}[t]
    \centerline{\includegraphics[width=0.7\linewidth]{figures/methodology/dcn_standard_vs_deformable.png}}
    \caption[Standard vs. deformable convolution]{Visualization of the difference between the fixed \imgpartdesc{a} and adaptive \imgpartdesc{b} receptive fields. Stacking multiple deformable convolutions results in profound amplification of deformation, making the transformation capture diverse shapes that would otherwise be very coarsely approximated by a standard convolution. \externalsrc{\cite{dai2017dcnn}}}
    \label{fig:StandardVsDeformableCNN}
\end{figure}
% ------------------------------------------------------------------------------

% ------------------------------------------------------------------------------
\begin{figure}[t]
    \centerline{\includegraphics[width=0.4\linewidth]{figures/methodology/deformable_convolution.pdf}}
    \caption[\Gls{dcnn}]{Illustration of a $3 \times 3$ deformable convolution operation. Unlike the standard convolution operation used in neural networks, this one employs one additional step of predicting variable offsets instead of using a fixed rectangular grid. \externalsrc{\cite{dai2017dcnn}}}
    \label{fig:DeformableCNN}
\end{figure}
% ------------------------------------------------------------------------------

% ------------------------------------------------------------------------------
\begin{figure}[t]
    \centerline{\includegraphics[width=0.7\linewidth]{figures/methodology/dcn_sampling_locations.png}}
    \caption[Various sampling locations in \glspl{dcnn}]{Deformable convolution is effective at learning appropriate sampling locations reflecting the underlying transformation. \imgpartdesc{a} shows the regular sampling grid of a standard convolution; \imgpartdesc{b} is an example of irregularly deformed sampling region; \imgpartdesc{c} and \imgpartdesc{d} represent an expected pattern corresponding to scaling and rotation operations, respectively. \externalsrc{\cite{dai2017dcnn}}}
    \label{fig:SamplingLocationsDeformableCNN}
\end{figure}
% ------------------------------------------------------------------------------

The original paper~\cite{dai2017dcnn}, where \glspl{dcnn} were introduced showed, that learning dense spatial transformation in using deep learning by use of \glspl{cnn} or sophisticated vision tasks such as object detection and semantic segmentation is feasible as well as effective. After our experience, we add that object tracking may benefit from this extension, too.
