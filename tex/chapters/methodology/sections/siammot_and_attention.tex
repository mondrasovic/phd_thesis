\section{Siamese Multi-Object Tracking and Attention}
\label{sec:SiamMOTandAttention}

% ##############################################################################
\subsection{Motivation}

% ##############################################################################
\subsection{Deformable Convolutional Neural Networks}
\label{ssec:DeformableCNNS}

\Glspl{dcnn}~\cite{dai2017dcnn} are gaining popularity and are being applied to numerous sophisticated computer vision tasks, \egtext{}, object segmentation (dense predictions) and object detection (semi-dense predictions). Since object tracking revolves around the same requirements in terms of pixel-wise precision, we contemplated using this advancement as well.

Although \glspl{cnn} (\sectiontext{}~\ref{ssec:ConvolutionalNeuralNetworks} on page~\pageref{ssec:ConvolutionalNeuralNetworks}) are an excellent tool for a plethora of deep learning tasks involving image processing, they are still limited in their capabilities to model a broad range geometric transformation. To address this, practictioners apply a broad range of data augmentation techniques (\egtext{}, rotation, translation, scaling, and shearing) to provide the necessary samples of some particular transformation during the training. However, such approach is limited to tailor-made transformations that may not cover the entire set of possibilities the model may face in practice. Also, there is a large body of literature on designing transformation invariant features~\cite{lowe1999sift, rublee2011orb}. We also encountered scale and rotation equivariant Siamese trackers that inherently handled the two transformations.

The first work to learn spatial transformation from the training data in a deep learning fashion is known under the name \glspl{stn}~\cite{jaderberg2016stn}. It warps the feature map via a global parametric transformation such as affine transformation

% ------------------------------------------------------------------------------
\begin{figure}[t]
    \centerline{\includegraphics[width=0.6\linewidth]{figures/methodology/deformable_convolution.pdf}}
    \caption[\Gls{dcnn}]{ \externalsrc{\cite{dai2017dcnn}}}
    \label{fig:DeformableCNN}
\end{figure}
% ------------------------------------------------------------------------------

% ------------------------------------------------------------------------------
\begin{figure}[t]
    \centerline{\includegraphics[width=0.6\linewidth]{figures/methodology/dcn_standard_vs_deformable.png}}
    \caption[Standard vs deformable convolution]{.\externalsrc{\cite{dai2017dcnn}}}
    \label{fig:StandardVsDeformableCNN}
\end{figure}
% ------------------------------------------------------------------------------

% ------------------------------------------------------------------------------
\begin{figure}[t]
    \centerline{\includegraphics[width=0.6\linewidth]{figures/methodology/dcn_sampling_locations.png}}
    \caption[Various sampling locations in \glspl{dcnn}]{.\externalsrc{\cite{dai2017dcnn}}}
    \label{fig:SamplingLocationsDeformableCNN}
\end{figure}
% ------------------------------------------------------------------------------
