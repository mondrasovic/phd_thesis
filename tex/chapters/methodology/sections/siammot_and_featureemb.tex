\section{Siamese Multi-Object Tracking and Feature Embedding}
\label{sec:SiamMOTandFeatureEmb}

\subsection{Motivation}

One of our experiments involved an end-to-end training of the whole \siammot{} architecture together with a custom head supposed to create embeddings based on \gls{roi}-pooled backbone-extracted features for the object \gls{bbox}. The motivation was to force the training process into extracting features that are not only satisfactory for detection and tracking, but also contain the necessary information to create embeddings for subsequent \gls{reid} purposes during the inference.

We strived for simplicity by extending the processing pipeline without altering the existing infrastructure. From the standpoint of implementation, the \siammot{} project itself is organized in a proper object-oriented and modular fashion, which made this particular development task easy in this aspect. However, as we will discuss further, we encountered setbacks in terms of training stability and we had to take appropriate measures. Furthermore, extending an already huge model that requires a significant amount of GPU VRAM made it even more demanding.

During the research related for our Siamese tracking survey~\cite{ondrasovic2021siamese}, we noticed one work where the exemplar features were projected using global average pooling operation into an embedding space of consisting of fewer dimensions~\cite{li2020figsiam}. The embedding vector was produced using the feature tensor that represents the kernel for the cross-correlation operation in the majority of the Siamese trackers discussed so far.

More concretely, suppose the extracted features were represented by a tensor of shape $8 \times 8 \times 256$. Then, the average pooling operation along the channel dimmension would produce a tensor of shape $1 \times 1 \times 256$, which can then be further flattened into a single $256$-dimensional vector. In the end, the obtained vector was $l_2$-normalized and thus projected onto a unit hypersphere. In the work of Li~\etal{}~\cite{li2020figsiam}, these embedding vectors were exploited for template updating and for combining multiple templates within a pool of size $n$ in an exponential fashion.

This observation led us to the following hypothesis. Given the fact the Siamese exemplar features do contain some, althougt probably not sufficient information for object \gls{reid}, would it be possible to map them further using a non-linear function to produce embedding vectors that could serve for \gls{reid}? Such features are just a learned template, therefore some notion of similarity needs to be already built into it.

\subsection{Training Phase}

\subsection{Inference Phase}

\subsubsection{Feature-based Non-Maximum Suppression}
\label{sssec:FeatureNonMaximumSuppression}

Salscheider~\cite{salscheider2020featurenms} proposed an extended \gls{nms} algorithm that incorporates a distance between feature embeddings dubbed as \featurenms{}. Considering our idea introduced above, we had to encompass the vector embeddings into the solver reasoning. At the beginning, we came up with the solution that exactly copied the one the mentioned author proposed. That provided further justification for attempting to implement the algorithm and test it in practice. The advantage is that this approach is restricted to the inference phase, thus experimenting with it did not require model re-training.

We assume the reader is acquainted with the original \gls{nms} algorithm (more in Section~\ref{ssec:NonMaximumSuppression}.). Nevertheless, here we repeat the same definitions for clarity. Let $\mset{B} = \cbrackets{\vect{b}_1, \vect{b}_2, \dots, \vect{b}_n}$ be a set of $n$ region proposals described by $n$ \glspl{bbox}. Scores for each detection are contained in a set $\mset{S} = \cbrackets{s_1, s_2, \dots, s_n}$, where $s_i$ denotes a detection score for the $i$-th box, $\vect{b_i}$. This time, we are also going to need the associated feature embedding vectors with each \gls{bbox}, represented in a set $\mset{E} = \cbrackets{\vect{e}_1, \vect{e}_2, \dots, \vect{e}_n}$. Let $\mset{B}_{nms}$ be  the set of filtered proposal instances from the set $\mset{B}$ produced using the \gls{nms} algorithm.

\def\threshlower{\tau_{\text{lower}}}
\def\threshupper{\tau_{\text{upper}}}
\def\threshsim{\delta}

The distinction in terms of parameters is the following. The original algorithm required only one threshold for the maximum allowed portion of the overlap between regions. Now, the reasonining is different. \featurenms{} requires three parameters discussed below.
\begin{itemize}
    \item A minimum threshold $\threshlower$ denoting the boundary below which the two objects are deemed as different. This value should be low, for example, $0.2$, which means that if the \gls{iou} between the two objects is less than $0.2$, then the two instances should be treated as different objects.
    \item A maximum threshold $\threshupper$ denoting a boundary above which the two objects are considered to be identical. Conversely to the $\threshlower$, this value should be high, for instance, $0.8$, which indicates that if the \gls{iou} of the two object instances surpasses this threshold, then it should be the same object, and thus, the \gls{bbox} with the lower confidence is discarded.
    \item A threshold $\threshsim$ used as a decision boundary between the embedding vectors. This threshold should reflect a measure of similarity. If the adopted measure of similarity (cosine distance, Euclidean distance, ...) falls below $\threshsim$, then the two objects are different, otherwise, they are considered the same one. This value of $\threshsim$ is used only if the two conditions above do not hold.
\end{itemize}

Here we provide a pseudocode of the extended \gls{nms} algorithm (\featurenms{}):

\begin{algorithmic}[1]
    \Function{Feature-NMS}{$\mset{B}$, $\mset{S}$, $\mset{E}$, $\threshlower$, $\threshupper$, $\threshsim$}

    \State $\mset{B}_{fnms}$ $\gets$ $\emptyset$
    \Comment{initialize the output (filtered) set of region proposals}

    \While {$\mset{B} \neq \emptyset$}
    \Comment{loop until all the proposals are processed}

    \State $m \gets \underset{i \in \cbrackets{1, 2, \dots, \msetsize{S}}}{\argmax{}} \mset{S}$
    \Comment{find an index of a proposal with the highest score}

    \State $\mset{B} \gets \mset{B} - \vect{b}_m$, $\mset{S} \gets \mset{S} - s_m$, $\mset{E} \gets \mset{E} - \vect{e}_m$
    \Comment{remove the proposal}

    \State $\mset{B}_{fnms} \gets \mset{B}_{fnms} \cup \vect{b}_m$
    \Comment{save the proposal with the highest score}

    \For{$i \gets 1$ to $\msetsize{B}$}
    \Comment{iterate through remaining proposals}

    \If{\Call{iou}{$\vect{b}_m$, $\vect{b}_i$} $\geq \threshlower$}
    \Comment{above the lower-bound threshold}

    \If{\Call{iou}{$\vect{b}_m$, $\vect{b}_i$} $\geq \threshupper$}
    \Comment{above the upper-bound threshold}

    \State $\mset{B} \gets \mset{B} - \vect{b}_i$, $\mset{S} \gets \mset{S} - s_i$, $\mset{E} \gets \mset{E} - \vect{e}_i$
    \Comment{remove the proposal}

    \Else

    \If{\Call{similarity}{$\vect{e}_m$, $\vect{e}_i$} $\geq \threshsim$}
    \Comment{similarity above threshold}
    \State $\mset{B} \gets \mset{B} - \vect{b}_i$, $\mset{S} \gets \mset{S} - s_i$, $\mset{E} \gets \mset{E} - \vect{e}_i$
    \Comment{remove the proposal}
    \EndIf

    \EndIf
    \EndIf
    \EndFor
    \EndWhile

    \State \Return $\mset{B}_{fnms}$
    \EndFunction
\end{algorithmic}

\subsection{Experimental Evaluation}

\subsection{Discussion}

We conjecture that our inability to improve the tracker performance was not particularly caused by the feature embedding itself. There is a recently published work Lu~\etal{}~\cite{lu2020retinetrack}, who introduced their \retinatrack{} tracker. This framework exploited the base visual object detector called a \retinanet{}~\cite{lin2018focal} and then added principially the same head as we did for the purpose of producing feature embeddings that could be used for \gls{reid}. However, there are obvious differences between the two trackers in terms of how the inference phase is executed, and that is where we see the root cause of our failure.
