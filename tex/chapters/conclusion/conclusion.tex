\chapter{Conclusion}
\label{chap:Conclusion}

The main objective of this dissertation thesis was to contribute to the field of \gls{vot} using the tools of deep machine learning (\chaptertext{}~\ref{chap:Goals},~p.~\pageref{chap:Goals}). At the beginning of this write-up, we introduced several concepts regarding object tracking, primarily when dealing with visual input in the form of a video. For the most part, we surveyed general object trackers, yet we intended to apply our developed methods to tracking vehicles, an important area with a vast practical impact. We have identified a plethora of approaches to tracking, but the most promising seems to be Siamese fully convolutional trackers (\sectiontext{}~\ref{sec:SingleObjectTracking},~p.~\pageref{sec:SingleObjectTracking}).

Siamese neural networks form the basis of the leading branch of trackers exploiting the properties of similarity learning (\sectiontext{}~\ref{ssec:LearningMetricEmbedding},~p.~\pageref{ssec:LearningMetricEmbedding}). This approach facilitates the creation of latent spaces with specific traits. The aim is to create a space where a trivial distance measurement between feature vectors of the embedded objects reflects their task-specific degree of similarity. In our case, such a similarity measure should be invariant to various distortions in lightning and object position as well as occlusion of varying severity. Among multiple problems hindering the performance of object tracking, we have identified the occlusion as the one we would focus on, making it a specific objective of this thesis.

Examination of the latent spaces, in particular object \gls{reid}, offered insight into the possibilities of embeddings. At the same time, we observed that occlusion is tightly coupled with a presence of similar interference, \ietext{}, distractors. To prevent the tracker from drifting to the background (whether semantic or not) in presence of partial occlusion, mechanisms based on attention~\cite{vaswani2017attention} have shown promising results. These aspects formed the basis for our experimentation, which utilized modern computer vision methods based on deep learning, primarily \glspl{cnn}~\cite{krizhevsky2012classification}.

This work provides a three-fold \textbf{contribution}, both theoretical and practical:
\begin{enumerate}
    \item We covered the recent advancements in the field of Siamese \gls{vot} in our up-to-date survey paper published in a Q$2$ journal~\cite{ondrasovic2021siamese}, filling the gap in the existing survey literature on this topic. This publication discusses the fundamental traits of Siamese trackers and the current problems they face. Both qualitative and quantitative comparison of the \gls{sota} approaches is presented, too (\sectiontext{}~\ref{sec:SingleObjectTracking},~p.~\ref{sec:SingleObjectTracking}).
    \item The vehicle tracking domain spurred the need for the removal of perspective distortion as we also demanded to measure vehicle speed and dimensions as part of our real-world applications. We exploited a homography to facilitate image rectification. We planned to incorporate homography mapping into the tracking itself, but due to the lack of available datasets, it could not be accomplished. Still, we developed an original approach to solving one specific use case when the homography can be exploited that earned us a Q$1$ journal publication~\cite{ondrasovic2021homography} (\chaptertext{}~\ref{chap:HomographyRanking},~p.~\pageref{chap:HomographyRanking}).
    \item Our practical contribution to the Siamese tracking consists of three parts, the last one being the most important for its improvements. We aimed at enhancing a \gls{sota} tracker called \gls{siammot}~\cite{shuai2021siammot}, using the \uadetrac{}~\cite{wen2020uadetrac} dataset.
          \begin{itemize}
              \item In our first experiment we showed the detrimental effects of \gls{reid} on the \gls{siammot} tracker due to the presence of object occlusion that causes the pollution of embedding vectors, rendering them useless for measuring object similarity. Our observations are pertinent to a general discussion regarding the use of \gls{reid} with Siamese trackers (\sectiontext{}~\ref{sec:SiamMOTandReID},~p.~\pageref{sec:SiamMOTandReID}).
              \item Our second experiment invovled extending the underlying \gls{siammot} architecture by another head that produced feature embeddings. The entire model was end-to-end trainable. Despite the lack of improvements, we explored ourselves the consequences of introducing embeddings into \gls{rpn}-based trackers, providing validation for conclusions from a relevant paper~\cite{zhang2021fairmot} concerning the ``unfairness'' of \gls{reid} in \gls{mot} (\sectiontext{}~\ref{sec:SiamMOTandFeatureEmb},~p.~\pageref{sec:SiamMOTandFeatureEmb}).
              \item In the third experiment, we attempted to enhance the discriminative ability of the tracker to handle scenes with the presence of partial occlusion. We developed an attention-based approach that produced promising results. Considering that, we adopted an already published architecture with our custom modifications that were proven to significantly improve Siamese-based \gls{sot}, namely the \gls{dsa}~\cite{yu2021dsa} module. As a result, we achieved a $2.6$\% improvement in the \gls{mota} metric (\sectiontext{}~\ref{sec:SiamMOTandAttention},~p.~\pageref{sec:SiamMOTandAttention}).
          \end{itemize}
\end{enumerate}

In conclusion, two major contributions are finished, namely the Siamese tracking survey and the homography ranking method, both of which have been successfully published. As for our attention-based extension, we see the potential for future work. This approach needs further validation on different benchmark datasets to compare our model with different trackers. However, traffic analysis was the domain we have focused on, and for that purpose, our evaluations showed positive results. Besides, there is a need for an extensive ablation study to determine the exact contribution of the attention to handling partial occlusion. For example, additional experimentation to quantify the correlation between the degree of partial occlusion and the improvement with the \gls{dsa} module itself.

Based on the knowledge we have acquired via studying Siamese tracking, our recommendation is the following. The inclusion of \gls{reid} to deal with object occlusion in Siamese trackers is very problematic since it would require sophisticated occlusion detection in the first place to avoid polluting the exemplar of the tracked object. Then, adding a head targeted at forming embeddings should be avoided in \gls{rpn}-based trackers for multiple reasons we explained in this work. The idea is, by all means, appropriate for tracking and it has been demonstrated to work~\cite{lu2020retinatrack, zhang2021fairmot}, however, not every architecture is suitable for such an extension. Last but not least, we have observed the attention mechanism to enhance trackers on multiple occasions~\cite{he2018twofoldsiam, wang2018learningattentions, yu2021dsa, li2018spatialawaresiam}, and our case was no exception. We think that devising a module that would require less \gls{gpu} \gls{vram} would broaden the potential for real-world applications. Attention has been incorporated into \gls{sot}, but in case of \gls{mot}, it is not so common. Considering the problems we encountered, we see why. Nevertheless, we think that Siamese tracking is yet to be fully explored, especially in the context of \gls{mot}, as there is a burgeoning demand for fast and accurate trackers.