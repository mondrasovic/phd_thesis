\chapter{Conclusion}
\label{chap:Conclusion}

% ==============================================================================
The main objective of this work was to summarize our research in the field of visual object tracking. We introduced several concepts regarding object tracking, primarily when dealing with visual input in the form of a video. We have identified a plethora of approaches, but the most promising seems to be Siamese-based fully convolutional trackers.

The concept of Siamese networks is the leading branch of trackers exploiting the properties of similarity learning. The approach of similarity learning facilitates the creation of metric spaces with specific traits. Without loss of generality, these traits are defined during the training by the data that is fed to the neural network model. The aim is to create a space where a trivial distance measurement between feature vectors of the embedded objects reflects their similarity. Such a similarity measure should be invariant to various distortions in lightning and object position, as well as occlusion of varying severity. Still, it should capture enough information to accurately indicate whether two objects are the same one or not. Among multiple problems hindering the performance of object tracking, we have identified the occlusion as the one we will focus on. We believe that the utilization of similarity learning should mitigate the consequences the occlusion has on the tracker's outcome, such as moving the attention from one object to a different one or losing the object completely.

For the most part, we surveyed general object trackers, yet our intention is to apply our work to vehicles. Even though we also researched vehicle tracking, the relative lack of detailed elaboration is evidence of the shortage of published research in that area. More than once when we mentioned that even face re-identification had been explored a lot more than vehicle re-identification. Taking this into consideration, we would venture to claim that vehicle tracking is still a largely unexplored area, too. We remark that it is an important area with a vast practical impact.

Investigation of general principles of object tracking has provided a foundation for what we can expect from the currently best object trackers. Examination of the concept of learning metric spaces, in particular, object re-identification, offered insight into the possibilities of task-specific embeddings. We believe that a combination of the two mentioned approaches could help us to fulfill the main goals of this work that focus on improving visual object tracking under the presence of occlusion using deep machine learning tools.