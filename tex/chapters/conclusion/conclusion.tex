\chapter{Conclusion}
\label{chap:Conclusion}

\todo[inline]{Finish conclusion.}

The main objective of this thesis was to contribute to the field of visual object tracking. Specifically, we aimed at improving a \gls{sota} multi-object tracker of our choice, the \gls{siammot} framework, for traffic analysis. The core of our contribution stands at enhancing the discriminative ability of the tracker to handle scenes where objects of interest become occluded. In terms of methodology, we heavily relied on deep machine learning and Siamese neural networks, which the \gls{siammot} tracker is built upon.

At the beginning of this write-up, we introduced several concepts regarding object tracking, primarily when dealing with visual input in the form of a video. We have identified a plethora of approaches, but the most promising seems to be Siamese-based fully convolutional trackers.

Siamese neural networks form the basis of the leading branch of trackers exploiting the properties of similarity learning. The approach of similarity learning facilitates the creation of metric spaces with specific traits. Without loss of generality, these traits are defined during the training by the data that is fed to the neural network model. The aim is to create a space where a trivial distance measurement between feature vectors of the embedded objects reflects their similarity. Such a similarity measure should be invariant to various distortions in lightning and object position, as well as occlusion of varying severity. Still, it should capture enough information to accurately indicate whether two objects are the same or not. Among multiple problems hindering the performance of object tracking, we have identified the occlusion as the one we would focus on.

For the most part, we surveyed general object trackers, yet we intend to apply our work to tracking vehicles. Even though we also researched vehicle tracking, the relative lack of detailed elaboration is evidence of the shortage of published research in that area. More than once when we mentioned that even face \gls{reid} had been explored a lot more than vehicle \gls{reid}. Taking this into consideration, we would venture to claim that vehicle tracking is still a largely unexplored area, too. We remark that it is an important area with a vast practical impact.

Investigation of general principles of object tracking has provided a foundation for what we can expect from the currently best object trackers. Examination of the concept of learning metric spaces, in particular, object \gls{reid}, offered insight into the possibilities of task-specific embeddings. At the same time, occlusion is tightly coupled with similar interference. To prevent the tracker from drifting to the background (whether semantic or not), mechanisms based on attention have shown promising results. Not only does our survey~\cite{ondrasovic2021siamese} cover such approaches explicitly in one section, our most relevant contribution to object tracking exploits the attention as well.
