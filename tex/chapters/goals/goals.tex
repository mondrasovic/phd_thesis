\chapter{Dissertation Thesis Goals}
\label{chap:Goals}

\Gls{vot} has been increasingly studied in recent years due to its real-world applications. The research area is vast, very active, and encompasses a broad range of approaches. Deep learning has occupied a great deal of the time that we spent studying and researching. The intention to build a working solution and incrementally advance the domain of object tracking has been the main incentive behind the choice for the topic of this dissertation thesis. With this in mind, \textbf{the general goal of this dissertation thesis is to improve the accuracy of visual object tracking using deep machine learning tools}.

Tracking of objects using visual features may produce the output in many forms, the usual axis-aligned \glspl{bbox}, rotated \glspl{bbox}~\cite{chen2019rotbboxes}, segmentation masks~\cite{wang2019siammask}, or even object contours~\cite{yang2016encoderdecoder}. We plan to primarily focus on solutions producing axis-aligned \glspl{bbox} since they are ubiquitous. We have mentioned our intention to deal with traffic-related scenarios. So far, there have been no restrictions as to why our tracker would have to run at real-time speed. Many of our real-world applications have involved camera-recorded videos that were processed offline. The decision to deal with traffic-related scenarios is also driven by projects supported by the University of Žilina. Companies of the private or public sector are interested in the automated analysis of traffic. Currently, a great deal of work such as vehicle counting is performed with human intervention. The author of this thesis as well as one of the supervisors were actively involved in solving problems related to tracking and traffic analysis using recorded videos during the research performed as part of this dissertation.

There are diverse factors that contribute to the overall performance of a tracker. Our analysis has led us to narrow our focus to occlusion handling. Occlusion can impose a huge precision penalty since the tracked object may be lost, or worse, the tracker's attention may be dragged away to a different object (a so-called drift problem). Many of the \gls{sota} solutions lack explicit occlusion handling~\cite{guo2019siamcar, li2018siamrpn, wang2019siammask} and we have identified this to be one of the leading causes of failure. Thus, \textbf{the specific goal is to propose a solution to problems regarding visual tracking that stem from the presence of occlusion, the transformation of the tracked objects as well as varying lighting conditions in the scene}.

Given what has been presented so far, we have chosen the path of similarity learning and a closely related \gls{reid} (\sectiontext{}~\ref{ssec:LearningMetricEmbedding}). Despite the existence of different techniques, we have been convinced during this initial research phase about the great potential of metric spaces and their properties that seem to fit our needs~\cite{liu2016ssd}. Some works argue that a well-built similarity function based upon metric learning in combination with a simple matching algorithm on the level of \glspl{bbox} can produce a reasonable performance~\cite{tao2016sint}. We think that even humans would be capable of discerning between objects when shown their pictures from distinct times in a video even dozens of seconds apart. The inherent visual clue about the object that makes it stand out among the set of others should be present most of the time. However, it may not be always possible to unambiguously identify an object given its appearance, especially when it comes to vehicles. Special marks such as customized paintings, decorations, or even scratches become relevant~\cite{liu2016ssd}. Consequently, the use of attention-based~\cite{vaswani2017attention} approaches has yielded promising results. In addition, attention may serve the purpose of enhancing the tracker's discriminating power when it comes to detecting objects undergoing a partial occlusion. Thus, expanding upon the foundation of the specific goal, we will strive to propose and implement a solution that can improve the performance of a tracker by considering the obstacles mentioned above. Therefore, \textbf{the goal from a methodological perspective is the application of approaches based on attention or similarity learning to handle object occlusion of varying intensity, change in the position and viewpoint of the tracked object, and fluctuations in the scene illumination}.
