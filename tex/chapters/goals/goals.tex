\chapter{Dissertation Thesis Goals}
\label{chap:Goals}

% =================================================================================================
\Gls{vot} is increasingly studied in recent years due to its real-world applications. Given the elaboration provided in the \cref{chap:TheoreticalFoundations}, we believe that investing research and development time into the topic of object tracking is worthwhile. The results presented so far are by no means everything that could be said about this topic. Quite to the contrary, the research area is vast, very active, and encompasses a broad range of approaches. Deep learning has occupied a great deal of the time that we spent studying and researching. The intention to build a working solution and incrementally advance the domain of object tracking has been the main incentive behind the choice for the topic of this dissertation thesis. With this in mind, \textbf{the general goal of this dissertation thesis is to improve the accuracy of visual object tracking using deep machine learning tools}.

Tracking of objects using visual features may produce the output in many forms, the usual axis-aligned \glspl{bbox}, rotated \glspl{bbox}~\cite{Chen2019}, segmentation masks~\cite{Wang2019}, or even object contours~\cite{Yang2016}. We plan to primarily focus on solutions producing axis-aligned \glspl{bbox}. Another variation is the speed at which the tracker processes the input. We have mentioned several times our intention to deal with traffic-related scenarios. So far, there have been no restrictions as to why our tracker would have to run at real-time speed. Many of our real-world applications have involved camera recorded videos that were processed offline. The decision to deal with traffic-related scenarios is also driven by projects supported by the University of Žilina. Companies of the private or public sector are interested in the automated analysis of traffic. Currently, a great deal of work such as vehicle counting is performed with human intervention. The author of this thesis as well as one of the supervisors are actively involved in solving problems related to tracking and traffic analysis using recorded videos.

There are diverse factors that contribute to the overall performance of a tracker. Our analysis has led us to narrow our focus to occlusion handling. Occlusion can impose a huge precision penalty since the tracked object may be lost, or worse, the tracker's attention may be dragged away to a different object. Current \gls{sota} solutions lack explicit occlusion handling~\cite{Guo2019, Li2018, Wang2019} and we have identified this to be one of the leading causes of failure. Thus, \textbf{the specific goal is to propose a solution to problems regarding visual tracking that stem from the presence of occlusion, the transformation of the tracked objects as well as varying lighting conditions in the scene}.

Given what has been presented so far, we have chosen the path of similarity learning and \gls{reid} (\cref{ssec:LearningMetricEmbedding}). Despite the existence of different techniques, we have been convinced during this initial research phase about the great potential of metric spaces and their properties that seem to fit our needs~\cite{Liu2016}. Our research of modern publications brought forward in the previous chapters points to the direction of embeddings and metric learning (\cref{sec:LatentSpacesAndEmbeddings}), specifically when applied to \gls{reid}. Besides the possibility to employ motion models for better prediction of locations in the absence of visual input, some research papers indicate that a well-built similarity function based upon metric learning in combination with a simple matching algorithm on the level of \glspl{bbox} can produce a reasonable performance~\cite{Tao2016}. We think that even humans would be capable of discerning between objects when shown their pictures from distinct times in a video even dozens of seconds apart. The inherent visual clue about the object that makes it stand out among the set of others should be present most of the time. Variation in lightning should not dramatically influence the outcome. However, it may not be always possible to unambiguously identify an object given its appearance, especially when it comes to vehicles. Special marks such as customized paintings, decorations, or even scratches become relevant~\cite{Liu2016}. In those situations, it is crucial to exploit temporal information. Expanding upon the foundation of the specific goal, we will strive to propose and implement a solution that can improve the performance of a tracker by considering the obstacles mentioned above. Therefore, \textbf{the goal from a methodological perspective is the application of approaches based on \gls{reid} and similarity learning to support the solution to problems caused by the object occlusion of varying intensity, change in the position, and the viewpoint of the tracked object, and fluctuations in the scene illumination}.

In order to accomplish the aforementioned goals, this dissertation thesis should follow the steps given below:
\begin{itemize}
    \item Identification of major problems that occur in the task of visual object tracking.
    \item Outline of requirements for datasets for training and evaluation of models along with a possibility of comparison with other approaches.
    \item Adequate data preprocessing, either as a modification of existing datasets or creating new ones for object tracking, particularly in traffic-related scenarios.
    \item Analysis of the current \gls{sota} approaches dealing with visual tracking of objects.
    \item Review of the modern methods of \gls{reid} based on similarity learning. Assessment of the contrastive-based and triplet-based training paradigms.
    \item The exploitation of recent advances in object \gls{reid} to handle object occlusion and dynamic conditions in terms of illuminance and object position. 
    \item Design and implementation of a method for visual object tracking adopting advances in deep machine learning.
    \item Evaluation of the implemented method with respect to official benchmarks.
    \item Summarization and dissemination of the achieved results followed by recommendations for their practical applications.
\end{itemize}
