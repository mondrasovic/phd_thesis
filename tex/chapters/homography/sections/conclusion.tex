\section{Conclusion}

In this homography-related subpart of our object tracking research, we proposed a method that builds on top of existing approaches for homography estimation that utilize existing point correspondences. The method is a systematic ranking of a set of homography matrices while exploiting the proposed score function to establish the order. Each homography in such a set belongs to a specific marker.

We consistently demonstrated that the proposed solution is robust in presence of noise in the point correspondences. These correspondences can be either algorithmically found using feature-matching algorithms (e.g., SIFT~\cite{lowel1999objrecognition}, SURF~\cite{bay2008speeded}) or annotated manually, but one has to keep in mind that even human annotations are often inaccurate. We also showed the robustness of our method to a varying number of markers and a change in shape.

Generally speaking, all the improvements at individual ranking positions steadily decreased, reaching $0$\% improvement at around $\nicefrac{2}{3}~m$, where $m$ is the number of markers. A practically applicable statement would be the following: ``the first half of ranked homographies yields a better reprojection compared to the baseline on average.''. The baseline performance was given by an average OpenCV~\cite{bradski2008learning} reprojection error under the assumption of no prior preference of specific markers, hence the random marker selection.

A practical advantage of our algorithm is that it is invariant to the underlying homography estimation method. It can, therefore, serve as an extension to all existing or future approaches that handle point correspondences, either as part of run time or a post-processing stage. Moreover, it is computationally very efficient, as it scales well with a quadratic complexity $\func{\Theta}{m^2}$ in the number of markers, which is usually a single-digit number.

The proposed homography ranking found a real-world application within our solution for the university-related Interreg SK-CZ project where we tackled the problem of tracking vehicles for the purpose of speed and dimension estimation. Homography mapping was a necessary part of our approach. However, we did not continue with this branch of research due to the lack of available datasets that we would require for a deep learning-based object tracking solution involving perspective projections.
