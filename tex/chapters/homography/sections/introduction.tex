\section{Introduction}
\label{sec:HomographyIntroduction}

This section is dedicated to one of our experiments that were not completely related to the \gls{vot} itself, yet we achieved an original scientific contribution in this area when exploring certain solutions that could potentially be applied to object tracking, especially traffic analysis. Even though we did not set out for homography-based object tracking (explained later) due to limitations of available datasets, still we would like to elaborate on our developed approach. The proposed method was fully described as well as scrupulously tested under difficult conditions. We wrote up the whole research process in a paper called \textbf{Homography Ranking Based on Multiple Groups of Point Correspondences}~\cite{ondrasovic2021homography}, published in journal \textbf{Sensors} (web:~\cite{websensors}), under the category \emph{Physical websensors}. In what follows, we provide a concise report of our research. For more information, we suggest the reader use the aforementioned article, even though a great deal of information is duplicated here. Moreover, our preliminary discussion of this possibility with the initial proposal of the solution can be found in our first paper dubbed \textbf{Foundations for homography estimation in presence of redundant point correspondencies}~\cite{ondrasovic2020foundations} published in \textbf{Mathematics in Science and Technologies} (web:~\cite{webmistconf}) conference.
