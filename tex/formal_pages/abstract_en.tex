\thispagestyle{empty}

\begin{center}
    \Large{\textbf{Abstract}}
\end{center}

\noindent ONDRAŠOVIČ Milan, Ing.: Visual Object Tracking Using Siamese Neural Networks.
    [Dissertation thesis] University of Žilina in Žilina. Faculty of Management Science and Informatics. Department of Mathematical Methods and Operations Research. - Thesis supervisor: doc. Mgr. Ondrej Šuch, PhD. - Žilina: FRI ŽU, 2022, 140 pages.

\

\noindent In this dissertation thesis, we disseminate the results concerning our research in visual object tracking using deep machine learning with an emphasis on traffic analysis. This treatise largely revolves around similarity learning, an area dominated by Siamese neural networks. Based on the review of modern approaches to tracking and their weaknesses, an object occlusion was selected as the problem to focus on. To address this, object re-identification was explored to tackle a complete occlusion, whereas to tackle partial occlusion, we adopted an attention mechanism. Latent spaces and embeddings were exploited to enhance the object ID propagation between frames. The attention-based approach is able to consistently improve a state-of-the-art architecture of our choice as measured by the established metrics for evaluating multi-object tracking frameworks, making our first contribution. In addition, there are two other contributions from this work. Specifically, an up-to-date comprehensive survey of Siamese-based visual object tracking and the development of a new homography ranking method, which was our attempt to aid the tracking process using a perspective transformation.

\noindent \textbf{Keywords}: visual object tracking, deep machine learning, Siamese neural networks, latent spaces, attention mechanism, homography, traffic analysis
