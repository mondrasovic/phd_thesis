\thispagestyle{empty}

\begin{center}
    \Large{\textbf{Abstrakt}}
\end{center}

\noindent ONDRAŠOVIČ Milan, Ing.: Vizuálne trasovanie objektov použitím siamských neurónových sietí.
[Dizertačná práca] Žilinská univerzita v Žiline. Fakulta riadenia a informatiky. Katedra matematických metód a operačnej analýzy. - Vedúci dizertačnej práce: doc. Mgr. Ondrej Šuch, PhD. - Žilina: FRI ŽU, 2022, 140 strán.

\

\noindent V tejto dizertačnej práci predstavíme výsledky nášho výskumu v rámci vizuálneho trasovania objektov použitím hlbokého strojového učenia so zameraním na analýzu dopravy. Táto rozprava sa vo veľkej miere opiera o podobnostné učenie, ktorému dominujú siamské neurónové siete. Na základe prieskumu moderných prístupov k trasovaniu spoločne s ich nedostatkami bolo prekrývanie objektov vybrané ako problém, na ktorý sa budeme zameriavať. Dôsledky úplného prekrytia bola snaha riešiť pomocou re-identifikácie objektov, pričom čiastočné prekrytie sme adresovali použitím mechanizmu pozornosti. Latentné a vnorené priestory boli využité pre zlepšenie propagácie ID objektu medzi snímkami. Spomínaný prístup založený na pozornosti je schopný konzistentne vylepšiť jednu z najlepších architektúr trasovačov objektov podľa nášho výberu, čo tvorí náš prvý príspevok. Evaluácia je postavená na ustanovených metrikách určených na vyhodnocovanie trasovania viacerých objektov. Naviac, naše dva ďalšie príspevky boli publikované v žurnáloch. Konkrétne sa jedná o aktuálny, hĺbkový prehľadový článok zameraný na siamské vizuálne trasovanie objektov a o vývoj novej metódy slúžiacej na posudzovanie kvality reprojekcie homografie, ktorá bola súčasťou nášho pokusu pomôcť procesu trasovania za použitia perspektívnej transformácie.

\noindent \textbf{Kľúčové slová}: vizuálne trasovanie objektov, hlboké strojové učenie, Siamské neurónové siete, latentné priestory, mechanizmus pozornosti, homografia, analýza dopravy
