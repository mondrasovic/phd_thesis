% Mathematical operators.
% ========================================
\DeclareMathOperator*{\argmax}{arg\,max}
\DeclareMathOperator*{\argmin}{arg\,min}

% Vectors
% ========================================
\def\vx{{\mathbf{x}}}
\def\vy{{\mathbf{y}}}
\def\vtheta{{\mathbf{\theta}}}

\newcommand{\vect}[1]{\mathbf{#1}}
\newcommand{\vectt}[2]{{\mathbf{#1}}^T = \left[ {#2} \right]}

\newcommand{\euclnorm}[1]{{\left\Vert {#1} \right\Vert}_2}
\newcommand{\frobnorm}[1]{{\left\Vert {#1} \right\Vert}_F}

% Matrices
% ========================================
\newcommand{\mtx}[1]{\mathbf{#1}}
\newcommand{\mtxsup}[2]{\bm{#1}^{#2}}
\newcommand{\mtxsub}[2]{\bm{#1}_{#2}}
\newcommand{\mtxsubsup}[3]{\bm{#1}_{#2}^{#3}}

\def\H{\mtx{H}}

% Brackets
% ========================================
\newcommand{\rbrackets}[1]{\left( {#1} \right)}
\newcommand{\cbrackets}[1]{\left\{ {#1} \right\}}
\newcommand{\sbrackets}[1]{\left[ {#1} \right]}
\newcommand{\pbrackets}[1]{\left\Vert {#1} \right\Vert}

% Upper or lower index with or without brackets.
% ========================================
\newcommand{\suprbrackets}[2]{{#1}^{\left( {#2} \right)}}
\newcommand{\subsup}[3]{{#1}_{#2}^{#3}}
\newcommand{\subsuprbrackets}[3]{{#1}_{#2}^{\left( {#3} \right)}}

% Functions.
% ========================================
\newcommand{\func}[2]{{#1} \left( #2 \right)}
\newcommand{\minf}[1]{\text{min} \left( #1 \right)}
\newcommand{\maxf}[1]{\text{max} \left( #1 \right)}

\def\scoref{\mathcal{F}}
\def\lossf{\mathcal{L}}

% Probability and statistics.
% ========================================
\newcommand{\prob}[1]{p \left( #1 \right)}
\newcommand{\probgiven}[2]{p \left( #1 \ | \ #2 \right)}

% Sets.
% ========================================
\newcommand{\mset}[1]{\mathcal{#1}}
\newcommand{\msetsize}[1]{\left\vert \mathcal{#1} \right\vert}

% Algorithms.
% =================================================================================================
\def\arraydef{{\text{\textbf{array}}}}
\def\scores{\boldsymbol{s}}
\def\sortres{\boldsymbol{\omega}}

\newcommand{\srcfuncname}[1]{\texttt{{#1}()}}

% Table-related.
% ========================================
\newcommand{\tblcolname}[1]{\textbf{#1}} % Table column name

% Abbreviations that appear in equations often.
% ========================================
\def\TP{{\text{\gls{tp}}}}
\def\TN{{\text{\gls{tn}}}}
\def\FP{{\text{\gls{fp}}}}
\def\FN{{\text{\gls{fn}}}}

\def\AP{{\text{\gls{ap}}}}
\def\MAP{{\text{\gls{map}}}}

\def\IOU{{\text{\gls{iou}}}}

% Names.
% ========================================
\newcommand{\modelname}[1]{\texttt{#1}} % A general name of a model (an entire network or just its part).
\newcommand{\datasetname}[1]{\texttt{#1}} % A general name of a dataset.

\def\siammot{\modelname{SiamMOT}}
\def\fasterrcnn{\modelname{Faster R-CNN}}

\def\uadetrac{\datasetname{UA-DETRAC}}
\def\mscoco{\datasetname{MS-COCO}}
\def\pascalvoc{\datasetname{PASCAL-VOC}}

% Related to external sources: figures, tables, etc.
% ========================================
\newcommand{\externalsrc}[1]{(\textit{source}: {#1})}

% Other
% =================================================================================================
% Image description.
\newcommand{\imgpartdesc}[1]{(#1)}

\def\boxplotimgwidth{0.75\linewidth}

\def\etal{\textit{et al.}}
