\section{Visual Object Tracking}

% -------------------------------------------------------------------------------------------------
\subsection{OTB 2013 and 2015}
\label{ssec:DatasetOTB1315}

\todo[inline]{Add OTB2013 and 2015 summary.}

% -------------------------------------------------------------------------------------------------
\subsection{GOT10k}
\label{ssec:DatasetGOT10k}

\todo[inline]{Add GOT10k summary.}

% -------------------------------------------------------------------------------------------------
\subsection{VOT 2019}
\label{ssec:DatasetVOT2019}

\todo[inline]{Talk about variations (years 2015, 2016, 2018, 2019).}

This prominent dataset (web: \cite{vot2019dataset}) is a benchmark in visual object tracking \cite{Kristan2019a}. It has a long history of development and annual challenges for the best visual tracker. All \datasetname{\gls{vot} 2019} datasets are available through the \gls{vot} toolkit. The pipeline for evaluating a tracker is automated to facilitate ease of use and a framework for researchers to allow an objective comparison with others. The dataset provides multiple video sequences where a single object is present under various conditions. For example, people running, a fish swimming behind corals, a vehicle driving in a city, and so forth. Each object is annotated by a single \gls{bbox} per each frame in which it appears.

% -------------------------------------------------------------------------------------------------
\subsection{KITTI Object Tracking}
\label{ssec:DatasetKITTIObjectTracking}

This object tracking benchmark \cite{Geiger2012CVPR} (web: \cite{kittiobjecttrackingdataset}) consists of $21$ training sequences and $29$ test sequences. Even though there have been labeled $8$ different classes, only the classes "car" and "Pedestrian" are evaluated in this benchmark, as only for those classes enough instances for a comprehensive evaluation have been labeled. Considering our potential traffic application, this fact does not represent a disadvantage. The goal in the object tracking task in this benchmark is to estimate object tracklets for the classes "car" and "pedestrian". Only $2D$ $0$-based, axis-aligned \glspl{bbox} in each image are evaluated.

\begin{figure}[t]
    \centerline{\includegraphics[width=\linewidth]{figures/datasets/kitti_object_tracking_sample.jpg}}
    \caption[\datasetname{KITTI Object Tracking} dataset]{A sample of the only two classes ("car" and "pedestrian") that are evaluated in the benchmark using the \datasetname{KITTI Object Tracking} dataset. \externalsrc{\cite{kittiobjecttrackingdataset}}}
    \label{fig:DatasetKITTIObjectTracking}
\end{figure}

% -------------------------------------------------------------------------------------------------
\subsection{UA-DETRAC}
\label{ssec:DatasetUADETRAC}

The most important benchmark dataset for our work is \datasetname{UA-DETRAC} \cite{CVIU_UA-DETRAC} (web: \cite{uadetracdataset}). To the best of our knowledge, this dataset the most favorably suits our needs of all surveyed datasets available. The primary reason is that is provides a plethora of traffic situations recorded using a static camera. This setup appropriately reflects the requirements of our goal, which is the analysis of traffic scenes using object tracking algorithms. This work provides high quality human-generated annotations with a lot of additional information about the captured vehicles, such as the intensity of their occlusion.
\datasetname{UA-DETRAC} is considered a challenging real-world multi-object detection and multi-object tracking benchmark. The dataset consists of $10$ hours of videos captured at $24$ different locations in China. The videos are recorded at $25$ \gls{fps}, with resolution of $960 \times 540$ pixels. There are more than $140\ 000$ frames and $8\ 250$ vehicles that are manually annotated, leading to a total of $1.21$ million labeled \glspl{bbox} of objects.

\def\uadetracfigsize{0.45}

\begin{figure}[t]
    \centering
    \begin{subfigure}[b]{\uadetracfigsize\textwidth}
        \centering
        \includegraphics[width=\textwidth]{figures/datasets/uadetrac_stats_vehicle_category.png}
        \caption[]{}
    \end{subfigure}
    \hfill
    \begin{subfigure}[b]{\uadetracfigsize\textwidth}
        \centering
        \includegraphics[width=\textwidth]{figures/datasets/uadetrac_stats_weather.png}
        \caption[]{}
    \end{subfigure}
    \hfill
    \begin{subfigure}[b]{\uadetracfigsize\textwidth}
        \centering
        \includegraphics[width=\textwidth]{figures/datasets/uadetrac_stats_scale.png}
        \caption[]{}
    \end{subfigure}
    \hfill
    \begin{subfigure}[b]{\uadetracfigsize\textwidth}
        \centering
        \includegraphics[width=\textwidth]{figures/datasets/uadetrac_stats_occlusion_ratio.png}
        \caption[]{}
    \end{subfigure}
    \caption[\datasetname{UA-DETRAC} dataset overview]{Summary statistics of the \datasetname{UA-DETRAC} dataset. Subfigure (a) shows the distribution of vehicle categories, one of \emph{car}, \emph{bus}, \emph{van} or \emph{other}; (b) shows the varying weather conditions belonging to either \emph{night}, \emph{sunny}, \emph{rainy} or \emph{cloudy}; (c) depicts the change in scale given by the square root of the \gls{bbox} pixel area; and (d) reflects the occlusion ratio throughout the dataset computed as the fraction of the vehicle \gls{bbox} being occluded . \externalsrc{\cite{uadetracdataset}}}
    \label{fig:UADETRACStats}
\end{figure}
